\documentclass[12pt, oneside]{article}

\usepackage[noxy, goodsyntax]{virginialake}
\usepackage{xcolor}
\usepackage{amsthm}
\usepackage{titlesec}

\titleformat{\section}[block]{\normalfont\bfseries\fontsize{1.3em}{1.3em}\selectfont}{\thesection}{1em}{}

\theoremstyle{plain}
\newtheorem{thm}{Theorem}[section] 
\newtheorem{prop}[thm]{Proposition}
\newtheorem{lem}[thm]{Lemma} 
\theoremstyle{definition}
\newtheorem{defn}[thm]{Definition}

\begin{document}

\section{System $\SLSE$}

\begin{defn}
\textit{Structures} in $\SLSE$ (Symmetric Logic of Sequenced Exponentials) contain countably many \textit{atoms}, denoted a,b,c,d,.. \cite{strassburger2003mell}
\end{defn}

\begin{thm}
Socrates is mortal.
\end{thm}

\begin{proof}
Socrates is a man.
All men are mortal.
Therefore Socrates is mortal.
\end{proof}

\section{Permutability of Rules}
Trust me bro

\section{Cut elimination for ???}

\[
\vlderivation{
\vlin{\sw}{}{S'\{?(U,[(U',V),?R],V'\}}{
\vlin{\pU}{}{S'\{?(U,[U', ?R],V, V')\}}{
\vlin{\wD }{}{S'(?(U,[U,?R]),!(V,V'))}{
\vlhy{S'(?(U,U'),!(V,V'))}
}}}}
\]

\bibliographystyle{plain}
\bibliography{references}

\end{document}